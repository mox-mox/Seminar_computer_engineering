
%{{{ Preamble

\documentclass[journal]{IEEEtran}

\hyphenation{op-tical net-works semi-conduc-tor}

%{{{ Packages


%\usepackage{cite}

% Better looking tables
%\usepackage{array}

% Some stuff mox needs:
\usepackage{graphicx}
\graphicspath{{../material/paper/}}

\usepackage{hyperref}
\hypersetup{pdftex=true, colorlinks=false, breaklinks=true}

%\usepackage[usenames,dvipsnames]{xcolor}
\usepackage{tkz-kiviat,numprint,fullpage}
\usetikzlibrary{arrows}

\usepackage{xcolor}
\usepackage{minted}

\usepackage{lipsum}
\usepackage{xspace}
%}}}


%{{{ New commands

%\makeatletter
%\newcommand*{\textoverline}[1]{$\overline{\hbox{#1}}\m@th$}
%\makeatother
%}}}

%{{{ Avebritations

%\newcommand*{\krste}[0]{Krste Asanovi\'c\xspace}
\def\krste/{Krste Asanovi\'c}
%}}}



%}}}






\begin{document}
 %\setlength{\parindent}{0mm}

%{{{ Title, author, abstract and keywords

\title{RISC-V --- Architecture and Interfaces\\The RocketChip}


\author{Moritz~N\"oltner-Augustin\\%
		University of Heidelberg, ZITI}
%	\IEEEcompsocitemizethanks{\IEEEcompsocthanksitem M. N\"oltner is enrolled at the University of Heidelberg and is currently pursuing a B.S. degree in applied informatics.\protect\\
%	E-mail: sh386@ix.urz.uni-heidelberg.de}%
%\thanks{Manuscript received January 6, 2015; revised January 12, 2015.}}%

% The paper headers
\markboth{Advanced Seminar ``Computer Engineering'', UNIVERSITY OF HEIDELBERG WT16/17}%
{Shell \MakeLowercase{\textit{et al.}}: RISC-V --- Architecture and Interfaces\\The RocketChip}

% use for special paper notices
%\IEEEspecialpapernotice{(Invited Paper)}

\maketitle

% As a general rule, do not put math, special symbols or citations
% in the abstract or keywords.
\begin{abstract}
This paper gives a short overview of the RocketChip system-on-chip generator and how to use it.
\end{abstract}

% Note that keywords are not normally used for peerreview papers.
\begin{IEEEkeywords}
RISC-V, RocketChip, Boom, SoC generator.
\end{IEEEkeywords}

%}}}

\section{Introduction}
\IEEEPARstart{C}{omputer} technology has seen the rise and fall of many instruction set architectures (ISAs) over time.
Proprietary ISAs not only lead to fragmentation of the CPU market, but are also vulnerable to extinction when their proprietor companies run into financial troubles.
Along with design complexity and licensing issues, these considerations led \krste/ et al.\ to decide upon creating a new and free ISA for their next round of research projects at the University of California Berkeley (UCB).
This fifth reduced instruction set ISA developed at UCB, called RISC-V, is --unlike its predecessors-- not only meant for teaching but also actual implementation.
With three supported word-widths (32, 64 and 128 bits), RISC-V is aimed at all possible computational environments ranging from small embedded systems up to full scale supercomputers.
To facilitate widespread adoption, the ISA is licensed permissively, allowing use for academic and commercial use in open- and closed-source designs free of charge and now, roughly 4 years after its inception, RISC-V is used in a number of roles:

\begin{itemize}
	\item The LowRISC project aims to become the ``linux of the hardware word''
	\item SiFive and Open-V are creating custom silicon products
	\item ETH Zurich and Università di Bologna cooperate to create a state-of-the-art parallel microcontroller
	\item IIT Madras creates a processor
	\item NVIDIA will use the RISC-V ISA for the replacement of their Falcon processor
	\item UCB uses RISC-V processors for research purposes as well as for teaching
\end{itemize}


\section{RocketChip}
The UCB currently developes three lines of RISC-V processors, Sodor, Rocket and Boom.
Sodor is a collection of simple processors for use in hardware lectures, Rocket is a scalar in-order processor for personal applications and Boom is superscalar out-of-order processor for high performance applications.
All three are written in Chisel (Creating Hardware in a Scala Embedded Language), a hardware description language --as the name implies-- embedded in Scala.
Rocket and Boom share a framework for creating an implementable system from the Chisel source code including 

















\section{Conclusion}
%Both \twi\ and SPI are very mature, while still up-to-date bus systems for low-speed, low-complexity interconnection of integrated circuits.
%With their wide acceptance and when used as intended, both \twi\ and SPI are recommendable and reliable bus systems.


\bigskip
%\vfill

\begin{thebibliography}{1}

	\bibitem{spi_mc68hc11a8}
		Datasheet of the Motorola MC68HC11A8 microcontroller describing the SPI bus.\\
		Last downloaded 2014--01--10\\
		\url{http://cache.freescale.com/files/microcontrollers/doc/data_sheet/MC68HC11A8.pdf}
		\medskip


\end{thebibliography}
\enlargethispage{-5in}
\end{document}
